Day 11\textsuperscript{th} November 2021

\section{Template}\label{template}

\subsection{Header 2}\label{header-2}

Some information about something

\section{Axioms}\label{axioms}

\subsection{Questions}\label{questions}

1) Events \(A\) and \(B\) are \emph{independent} if

\[P(A \cap B) = P(A)P(B)\]

Give an example of independent Events illustrated as marbles in a finite
sample space

2) Consider the experiment of picking a random point in the unit square

\[R = \{(x,y)\ :\ 0 < x < 1,\ 0 < y < 1\}\]

where the probability of the point being in any particular region
contained within \(R\) is the area of that region, e.g., \(P(A) = ab\),
for a rectangle with sides length \(a\) and \(b\). Let \(A_{1}\) and
\(B_{1}\) be rectangles contained within \(R\), with areas not equal to
0 or 1. Let \(A\) be the event that the random point is in \(A_{1}\),
and \(B\) be the event that the random point is in \(B_{1}\).

a) Give a geometric description of when it is true that \(A\) and \(B\)
are independent and another example when they aren't independent.

b) Give a geometric description of conditional probability and show that
independence implies that

\[P\left( A \middle| B \right) = P(A)\]

c) Show that if \(A\) and \(B\) are independent, then

\[P(A \cup B) = P(A) + P(B) - P(A)P(B) = 1 - P\left( A^{c} \right)P(B^{c})\]

\subsection{Solutions}\label{solutions}

1) The sequence of 2 ordered coin-flips example can be used as a
solution. The Events ``\emph{the first flip is Heads''} and ``\emph{the
second flip is Tails''} are independent. S

For example, the Events \(A_{HH}\) and \(B_{TT}\) are disjoint, but not
independent, since
\(P\left( A_{HH} \right) = P\left( B_{TT} \right) = 1/4\), but
\(P\left( A_{HH} \cap B_{TT} \right) = 1/2 \neq 1/16\).

\section{Bibliography}\label{bibliography}

\begin{longtable}[]{@{}
  >{\raggedright\arraybackslash}p{(\columnwidth - 2\tabcolsep) * \real{0.0357}}
  >{\raggedright\arraybackslash}p{(\columnwidth - 2\tabcolsep) * \real{0.9643}}@{}}
\toprule\noalign{}
\begin{minipage}[b]{\linewidth}\raggedright
{[}1{]}
\end{minipage} & \begin{minipage}[b]{\linewidth}\raggedright
W. Feller, An Introduction to Probability Theory and Its Applications,
Vol. 1, 3rd Edition, 3rd ed., Wiley, 1968.
\end{minipage} \\
\midrule\noalign{}
\endhead
\bottomrule\noalign{}
\endlastfoot
{[}2{]} & W. Feller, An Introduction to Probability Theory and Its
Applications, Vol. 2, 2nd Edition, 2nd ed., John Wiley \& Sons, Inc.,
1971. \\
\end{longtable}
